\documentclass{amsart}
\usepackage{itaca}

\usepackage{minted, booktabs}\usemintedstyle{manni}

\begin{document}
La formula magica che permette di definire macro per tutti gli alfabeti in un attimo:
\begin{minted}[fontsize=\footnotesize]{latex}
% magia di latex3
\ExplSyntaxOn
\NewDocumentCommand{\makeabbrev}{mmm}
  {
   \yoruk_makeabbrev:nnn { #1 } { #2 } { #3 }
  }

\cs_new_protected:Npn \yoruk_makeabbrev:nnn #1 #2 #3
  {
   \clist_map_inline:nn { #3 }
    {
     \cs_new_protected:cpn { #2 } { #1 { ##1 } }
    }
  }
\ExplSyntaxOff

\DeclareMathAlphabet{\mybb}{U}{BOONDOX-ds}{m}{n}

\makeabbrev{\textbf}{bf#1}{
  a,b,c,d,e,g,h,i,j,k,l,m,n,o,p,q,r,t,u,v,w,x,y,z,
  A,B,C,D,E,G,H,I,J,K,L,M,N,O,P,Q,R,T,U,V,W,X,Y,Z }
\makeabbrev{\boldsymbol}{bs#1}{ <...> }
\makeabbrev{\mathsf}{sf#1}{ <...> }
\makeabbrev{\mathfrak}{fk#1}{ <...> }
\makeabbrev{\mathcal}{cl#1}{ <...> }
\makeabbrev{\mathbb}{bb#1}{ <...> }
\makeabbrev{\mybb}{mb#1}{ <...> }
\makeabbrev{\underline}{ul#1}{ <...> }
\makeabbrev{\overline}{ol#1}{ <...> }
  \end{minted}
In questo modo, si ha che
\begin{table}[h]
  \begin{center}
    \begin{tabular}{lcr}
      \verb|\bfA| \dots \verb|\bfZ|   & results in & $\bfA \dots \bfZ$ \\ \midrule
      \verb|\bsA| \dots \verb|\bsZ|   & results in & $\bsA \dots \bsZ$ \\ \midrule
      \verb|\sfA| \dots \verb|\fZ|   & results in & $\sfA \dots \sfZ$ \\ \midrule
      \verb|\fkA| \dots \verb|\fkZ|   & results in & $\fkA \dots \fkZ$ \\ \midrule
      \verb|\clA| \dots \verb|\clZ|  & results in & $\clA \dots \clZ$ \\ \midrule
      \verb|\bbA| \dots \verb|\bbZ| & results in & $\bbA \dots \bbZ$ \\ \midrule
      \verb|\mbA| \dots \verb|\mbZ| & results in & $\mbA \dots \mbZ$ \\ \midrule
      \verb|\ulA| \dots \verb|\ulZ| & results in & $\ulA \dots \ulZ$ \\ \midrule
      \verb|\olA| \dots \verb|\olZ| & results in & $\olA \dots \olZ$ \\ \bottomrule
    \end{tabular}
  \end{center}
\end{table}
\begin{minted}[fontsize=\small]{latex}
  \newlength{\seplen}
  \setlength{\seplen}{5pt}
  \newlength{\sepwid}
  \setlength{\sepwid}{.4pt}
  \def\firstblank{\,\rule{\seplen}{\sepwid}\,}
  \def\secondblank{\firstblank\llap{\raisebox{2pt}{\firstblank}}}
\end{minted}
\section*{first e secondblank}
Questo serve a disegnare i `blank' o `waiting for':
\begin{minted}[fontsize=\small]{latex}
  F(\firstblank, \secondblank) : \clC^\op \times \clC \to \clD
\end{minted}
\[F(\firstblank, \secondblank) : \clC^\op \times \clC \to \clD\notag\]
\section*{enumtag}
Questo serve a generare enumerate lievemente più belli:
\begin{minted}[fontsize=\small]{latex}
  \DeclareDocumentEnvironment{enumtag}{m}{
    \begin{enumerate}
      [ label = \textsc{#1}\oldstylenums{\arabic*}),
        ref   = \textsc{#1}\oldstylenums{\arabic*}
      ] }{ \end{enumerate} }
      %===
\begin{enumtag}{e}
  \item uno
  \item due
  \item tre
\end{enumtag}       
    \end{minted}
\begin{enumtag}{e}
  \item uno
  \item due
  \item tre
\end{enumtag}
\section*{smallmatrici}
Notare il diverso comportamento di \verb|\var| a seconda della presenza di un argomento opzionale:
\begin{minted}[fontsize=\small]{latex}
  \newenvironment{xsmallmatrix}[1]
    {\renewcommand\thickspace{\kern#1}\smallmatrix}
    {\endsmallmatrix}

  \NewDocumentCommand{\var}{o m m}{
    \IfNoValueTF{#1}{
      \left[
        \begin{smallmatrix}
          #2 \\
          \downarrow \\
          #3
        \end{smallmatrix}\right]}
    {
      \left[
        \begin{xsmallmatrix}{0em}
          & #2 \\
          #1 & \downarrow \\
          & #3
        \end{xsmallmatrix}\right]
    }
  }
\end{minted}
\[\var[f]{A}{B} \qquad \var{A}{B}\notag\]
\begin{minted}[fontsize=\footnotesize]{latex}
\var[f]{A}{B} \qquad \var{A}{B}
\end{minted}
\end{document}